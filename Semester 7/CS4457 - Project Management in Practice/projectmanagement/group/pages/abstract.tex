\begin{abstract}
	\begin{multicols*}{2}
		Group policy a term more commonly associated with the Windows Active Directory domain is defined as a set of rules 
		that govern an environments user and computer accounts.  Group policy provides a means of centralizing the 
		management and therein the control of the configuration; of client operating systems and their features.  
		\newline
		\newline
		Centralized management in a Linux environment is seen as a far more difficult problem.  
		The environment given the multitude of different distributions, all subscribing to their own 
		implementation philosophies have brought about almost infinite diversity, requiring companies to 
		employ highly skilled technicians to manage these ever changing environments.  
		These distributions or ``flavors'' seen to be highly different, however employ the same characteristics 
		in their fundamental implementation only differing in the tools provided to control these characteristics.  
		\newline
		\newline
		Since the conception of X.500 Directory Services specification eventually leading to Microsoft’s dominance in the
	    enterprise management, there has been a continual drive to integrate mixed environments into Active Directory.  
		However due to the mixed philosophies and therein the different distributions of Open Source Software (OSS) 
		operating systems, creating policies within active directory that can manage this diversity is seen as next to 
		impossible or at least limited.  
		\newline
		\newline
		Unlike Microsoft clients that implement a common integrating architecture this cannot be said for 
		Open Source Software (OSS) systems.  
		\columnbreak
		Enterprise solutions targeting systems such as Redhat Enterprise Linux and Suse Enterprise Linux aid 
		production administrators in provisioning and maintainability but not subscribe to non - production environments 
		where a greater degree of flexibility is expected within the environment or do they employ user friendly common 
		language as seen in the windows domain.  
		\newline
		\newline
		Observations within the field have led to the rapid development of Samba 4 offering directory services allowing 
		for Windows Policies to be integrated into the internal directory structure but as of yet still no viable solution 
		for Linux polices on the horizon.
		\newline
		\newline
		The scope of this document is to provide insight into the issues surrounding this IT domain, to provide an 
		analysis of the two main problems.  
		\newline
		\newline
		{\bf 1)} The demand for technicians to be knowledgeable in all the varying flavors of ‘Gnu’s not Unix’ (GNU) 
		systems and the abatement of this issue via common language, a domain specific language (DSL). 
		\newline
		\newline
		{\bf 2)} How these varying systems and their configurations can be represented by means of a directory services schema. 
		Like that of its Windows Group Policy Object (GPO) counterpart, the configuration, changeability and extensibility 
		of these policies through the use of the schema, should be applicable to all these systems. 
    \end{multicols*}
\end{abstract}
\cleardoublepage


