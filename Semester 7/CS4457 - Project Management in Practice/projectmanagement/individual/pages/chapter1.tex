\chapter{Overview}
\setcounter{page}{1}
	\section{Overview of the Final Year Project}
	    \begin{multicols}{2}
		\paragraph{}
			\vspace{5mm}
			Throughout this over view I will ease the reader into the domain concepts by providing the relationships with the Windows world.
			However as the overview progresses less comparisons will be made.  As really we are mixing our apples and oranges.  
			\newline
			\newline
			Group policy in the windows world provides administrators with an easy interface to control aspects of computer policies in an 
			easy defined manner.  Any computer joined to the domain is subject to these policies.  Administrators can, with the aid of visual
			snap-ins for Active directory, modify key-value pairs, which represent all the different aspects of a windows computer.  
			Furthermore with the interoperability or backwards compatibility built in, each successive release of windows conforms to 
			the standards or provides a translation pair relevant that that specific version of windows. 
			\index{Active Directory} 
			\index{Group policy}
			\index{Policies} 
			\index{Domain} 
			\index{key-value pairs} 
			\index{Interoperability} 
			\newline
			\newline
			Before the advent of the Windows Registry configurations for programs where kept in INI files, broken up into sections and properties.
			\index{Registry} 
			\index{INI files}
			\newline
			\newline
			[section]
			\newline
			property=value
			\newline
			\newline
			As the complexity of vendor applications and the operating system as a whole grew, so did the size of these INI files.  
			Furthermore for interoperability and the sharing of dynamic link libraries, which depended on these INI configurations, 
			it was quickly realized that this was an inefficient manner of storing configurations.  The Windows Registry solved this issue 
			by centralizing configurations settings into a hierarchical database containing settings for low-level operating system settings 
			as well as settings for applications running on the platform.  
			\index{Interoperability} 
			\index{INI files} 
			\index{Registry} 
			\index{Dynamic link libraries}
			\newline
			\newline
			Now that Windows has a central place for settings on the local machine, this provided an interface for a server (domain controller) 
			to apply settings to groups of machines also known as Group Policy. 
			\index{Group Policy} 
			\index{Domain controller} 
			\newline
			\newline
			Given this brief overview of Windows Group Policy, lets take a look at the ``Gnu’s not Unix'' (GNU) systems. 
			Since to conception of Linux in 1992 and the accompanied ‘Gnu’s not Unix’ (GNU) applications, file configuration settings still
		    remains the de facto way of configuring these systems and their application preferences.   The style of these configuration files 
		    is somewhat similar to that of INI files in that there is key/value pairs; differing in how comments are written.
			\index{Group Policy} 
			\index{INI files}
			\newline
			\newline
			As Linux grow in popularity in the business sector for backend main frames due to its stability and security, a need was required 
			for a centralized management of all these machines and a common login infrastructure to allow users and administrators to have
			credentials common to the network as a whole.  Yellow Pages (YP) also known as network infrastructure services (NIS) offered 
			this client server distributed system of authenticating users on a network.  
			\index{Yellow pages} 
			\index{Network infrastructure services}
			\newline
			\newline
			Configuration data compiling of user and group information along with hosts on the domain name system (DNS) domain allow for 
			this seamless user interaction between computers, but did not do much in allowing for administrators to manage these machines 
			in the central location.
			\index{Domain name system} 
			\newline
			\newline
			At this point I think it's important to look at the word ``domain'' as it will be used in contextually and comparatively, 
			extensively throughout this document.
			\newline
			\newline
			``The Domain Name System (DNS) is a hierarchical, distributed database that contains mappings of DNS domain names to various types 
			of data, such as Internet Protocol (IP) addresses. DNS allows you to use friendly names, such as www.microsoft.com, to easily 
			locate computers and other resources on a TCP/IP-based network. DNS is an Internet Engineering Task Force (IETF) standard.''
			\newline
			\newline
			``A Windows is a collection of computers in a networked environment that share a common database, directory database, or tree. 
			A domain is administered as a unit with common rules and procedures, which can include security policies, and each domain has 
			a unique name.''
			\newline
			\newline
			``A NIS ( Linux ) domain is similar to the Windows NT® domain system; although the internal implementation of the two are not 
			at all similar, the basic functionality can be compared.''
			\newline
			\newline
			``A domain as a field of scope or activity comprised of a specific knowledge set.'' 
			\newline
			\newline
			With reference to the definition of a windows domain it is important as a constitute part of this report to acknowledge the 
			concept of a domain as a group of computers.  Although this may create ambiguity and defer from the scope of the application, 
			it is however prominent to the concept and to that of the business terminology.  The term DNS domain or domain name will be 
			used as a reference to the identification label that defines an address, more commonly associated with the web in the form of 
			uniform resource locator (URL). And of course the term domain, by itself a reference a set of specific knowledge.
			\newline
			\newline
			Moving on from these definitions lets take a look at the problem domain.  Each individual major version of Linux distribution 
			provides utilities specific to controlling that machine and the settings therein.  These utilities in some cases modify file 
			configurations as previously discussed, For example, software provisioning is provided via ``yum'' on Redhat systems and ``apt'' 
			on Debian systems. Furthermore Redhat provides the tools ``chkconfig'' and ``service'' in controlling boot up configuration and 
			instant control respectively, while Debian provides similar tools.  As we start to compare the major distributions we start 
			to see the contrasting yet similar disparate natures of the utilities provided by the vendors.  
			\newline
			\newline
			Even though the underlying well-worn technologies that provide the backend implementation are primarily the same.  
			The tools provided to control and implement changes on these disjoint systems although comparatively different from a usage 
			perspective provide the same functionality.  This brings about the need for extremely skilled well-versed technicians and of 
			course creates more work from an administrative point of view.
			\newline
			\newline
			Given the success of group policy in the windows domain this seems like a logical candidate in tackling these systems as a whole.  
			By providing a framework to manage these contrasting systems through the use of a domain specific language; theoretically, an 
			abstraction layer of the problem domain could be modeled mitigating administrators to be learned in a plethora of 
			distribution specific commands.  
			Martin Fowler makes this argument from a contrasting point of view in terms of Language Oriented programming, by replacing a few 
			general purpose programming languages with many domain specific languages, he hypothesis that the requirement to learn numerous 
			application programmers interfaces can be more of a burden than learning a domain specific language catered for an individual task.
			\newline
			\newline
			Given this proposition for an abstraction layer to provide domain specific commands to a heterogeneous Linux environment and the 
			need for controlling these systems from a group policy perspective, a domain specific language seems an obvious candidate in 
			delivering a solution.  We will look further at this concept in the following sub sections of chapter one where we define the scope, 
			the objectives and terminology associated with this motive.
			\newline
			\newline
			Now that we have an overview of what the problem domain is and a vague idea of how it can be tackled, lets contrive the other 
			components.  Firstly the Client Server Model envisioned by the idea of a central authority, The domain specific language as the 
			intermediate language to provide instructions to the client, the schema for the specification of the database; where polices will 
			be created and stored and finally the administrative server front end which will be an over view of the application and how it 
			should enable administrators to create an modify policies for the underlying components.
	    \end{multicols}		
	