Appendices
\newline

http://www.prasannatech.net/2008/07/socket-programming-tutorial.html
Domain Specific Languages
\newline

style of programing ( language oriented programming ) Sergey Dmitrov
\newline

general purpose programming languages
\newline

small and limited languages ( dsls )
\newline

unix languages, lisp, macros
\newline

xml
\newline

data in text file, each line is an event, event data
\newline

parse event data an turn into usefull events
\newline

fuzzy distinction ( within the language )
\newline

attitue to design
\newline

external
\newline


-separate to the host language
\newline


-needs a compiler / interpreter to execute
\newline


-tied to base language
\newline


-awkward with mainstream { languages }
\newline

internal
\newline


-written in the host language
\newline


-conventional use of host language syntax
\newline


-Lack of Symbolic Integration
\newline


-complex parser / generator technologies
\newline


-Ignorant IDEs
\newline


-language cacophany
\newline

custom language to understand complex instructions
\newline

you have to problem anyway
\newline

DSL will provide simpler solution Patterns
\newline

\url{http://en.wikipedia.org/wiki/You_ain't_gonna_need_it}
\newline

\url{http://www.dofactory.com/Patterns/PatternStrategy.aspx}
\newline

\url{http://www.perl.com/pub/2003/08/07/design2.html}
Perl Filters 
\newline

\url{http://linux.about.com/library/cmd/blcmdl1_perlfilter.htm}    
\newline

\url{http://www.coderaptors.com/?Perl_Source_Filters}
\newline

\url{http://www.kichwa.com/quik_ref/spec_variables.html}
\newline

Domain specific languages are not a new concept.  They have been around since unix and lisp.
\newline

generic and specific approaches
\newline

generic approach is sub optimal
\newline

dichotomy - A division or contrast between two things that are or are represented as being opposed or entirely different.
\newline

dedicated language for solving problem in specific 
\newline

business processing, numeric computation and symbolic processing
\newline

evolving into general purpose languages
\newline

resurfaced of DSL's to solve domain specific problems resurfacing
\newline

subroutines to handle domain problems
\newline

OO frameworks with application specific code
\newline

Arie van Deursen Paul Klint Joost Visser 
"A domain-specific language (DSL) is a small, usually
declarative, language that offers expressive power focused
on a particular problem domain. In many cases,
DSL programs are translated to calls to a common subroutine
library and the DSL can be viewed as a means to
hide the details of that library."
\newline

semantic study of DSLS
\newline

terminology
\newline

risks and opportunities
\newline

example DSLs
\newline

DSL design methodology and DSL implementation strategies
\newline

importance of DSLS
\newline

Arie van Deursen Paul Klint Joost Visser 
"A domain-specific language (DSL) is a programming
language or executable specification language
that offers, through appropriate notations and abstractions,
expressive power focused on, and usually
restricted to, a particular problem domain."
\newline

expressive power
\newline

vagueness of the problem domain
\newline

domain modelling
\url{http://www.program-transformation.org/Transform/OrganizationDomainModeling}
\newline

DSLs are usually small, offering only a restricted suite of
notations and abstractions
\newline

Domain specific expressive power with Power of embedded General purpose language , known as embedded languages ( fowler alternative )
\newline

power restricted to problem domain
\newline

declarative seen as specific languages
\newline

DSL compiler known as application generator ( make reference to perl filter and how admin is not required to learn a further technology by use of a transformation into the intended language in which is was design ed and or implemented from )
\newline

application specific language
\newline

4th generation language ( 4gl) business data processing systems
\newline

balance between risk and opportunities. 
argue here that providing a filter as opposed to a compiler it allows the admin to take off from where the specification left off
\newline

extensibility, allow for the developed of the DSL, as the domain is user expressed
\newline

enhance productivity, reliability, maintainability, portability
\newline

"Domain-specific languages (DSLs) have the potential
to make software maintenance simpler: domain-experts
can directly use the DSL to make required routine modifications.
At the negative side, however, more substantial
changes may become more difficult: such changes
may involve altering the domain-specific language. This
will require compiler technology knowledge, which not
every commercial enterprise has easily available. The
paper describes and uses the experience of the RISLA
language for interest rate products to discuss the role of
DSLs in software maintenance, the opportunities introduced
by using them, and techniques for controlling the
risks involved."
\newline

argue here that providing a GPL abstract with  transformation engine, expressing that fowler said 'what the bloddy hell that lecturer was on about back in college about compilers, parse tress and grammer'
\newline

R. B. Kieburtz, L. McKinney, J. M. Bell, J. Hook,
A. Kotov, J. Lewis, D. P. Oliva, T. Sheard, I. Smith,
and L. Walton. A software engineering experiment in
software component generation. In Proceedings of the
18th International Conference on Software Engineering
ICSE-18, pages 542–553. IEEE, 1996.
Reports the results of an experiment in which a templatebased
approach and a DSL approach to software generation
were compared. Several subjects were monitored
while performing a number of development and maintenance
tasks using alternatively template technology and
DSL technology. Flexibility, productivity, reliability, and
usability were measured. The DSL approach scored better
on all counts.
\newline

much like group policy providing registry updates, dsl will provide optimzation at the domain levels
\newline

cost of maintaining the DSL. constructsa of the language should be module specific but follow general over flow and try to integrate resuability and common words.
\newline

reason for this that as the domain changes as it invariably does in the linux env, so will have the DSL and will incur over head of modifying the DSL as a whole.  Common terms should implment the same behaviour however specific terminonaly to a specific problem should be component based
\newline

cost of educating 
\newline

limited availability
\newline

scoping the DSL
\newline

balance between GPL and DSL  constructs
\newline

C. W. Krueger. Software reuse. ACM Computing Surveys,
24(2):131–183, June 1992.
\newline

Categorizes, describes and compares existing approaches
to software reuse, among which DSLs (or
application generators). Compared to the other approaches
DSLs reduce the intellectual effort required to
obtain an executable system from its specification. Limited
availability and difficulty of building DSLs of optimal
specificity/generality are listed as disadvantages of
DSLs.
\newline
T
he potential loss of efficiency when compared with
hand-coded software.
\newline

possibly talk about the proficiency of PERL as a text processor
\newline

examples of DSLS
\newline

pic, scatterm chem, lex, yacc and make 
SQL, BNF and html, xml
\newline

behavior and control
control and coordination [9, 10]
F. Bertrand and M. Augeraud. BDL: A specialized language
for per-object reactive control.
\newline

Design Methodology

Analysis (1) 
Identify the problem domain. (2) Gather all relevant
knowledge in this domain. (3) Cluster this knowledge
in a handful of semantic notions and operations on
them. (4) Design a DSL that concisely describes applications
in the domain.

Implementation (5) Construct a library that implements the
semantic notions. (6) Design and implement a compiler
that translates DSL programs to a sequence of library
calls.

Use (7) Write DSL programs for all desired applications and
compile them


DSL Implementation

Interpretation or compilation
Embedded languages / domain-specific libraries
Preprocessing or macro processing
Extensible compiler or interpreter

Apart from building a dedicated DSL compiler or interpreter,
or reusing the implementation of an underlying base
language, other implementation techniques may be used. For
instance, in aspect-oriented programming [46] a DSL is used
to describe an aspect of a system’s behavior that is orthogonal
to its main functionality. An aspect weaver is then used
to generate domain-specific code and merge it with the main
code.


Programming language technology
\newline

\url{http://en.wikipedia.org/wiki/Eric_Schmidt}
\newline

\url{http://en.wikipedia.org/wiki/Context-free_grammar}
\newline

\url{http://en.wikipedia.org/wiki/Metasyntax}
\newline

\url{http://en.wikipedia.org/wiki/Bash_(Unix_shell)}
\newline

\url{http://en.wikipedia.org/wiki/Yacc}
\newline

\url{http://en.wikipedia.org/wiki/GNU_bison}
\newline

\url{http://en.wikipedia.org/wiki/Lex_programming_tool}
\newline

\url{http://en.wikipedia.org/wiki/Flex_lexical_analyser}
\newline

\url{http://en.wikipedia.org/wiki/Lexical_analysis}
\newline

\url{http://en.wikipedia.org/wiki/Backus-Naur_form}
\newline

\url{http://en.wikipedia.org/wiki/Reentrant_(subroutine)}
\newline

\url{http://en.wikipedia.org/wiki/SLR_grammar}
\newline

\url{http://en.wikipedia.org/wiki/LL_parser}
\newline

\url{http://en.wikipedia.org/wiki/GLR_parser}
\newline

\url{http://en.wikipedia.org/wiki/Finite-state_machine}
\newline

\url{http://en.wikipedia.org/wiki/LR_parser}
\newline

\url{http://en.wikipedia.org/wiki/LALR}
\newline

How to Write a Simple Parser by Ashim Gupta
\newline

\url{http://ashimg.tripod.com/Parser.html}
\newline

\url{http://hivearchive.com/2007/06/10/regular-expressions-lisp-sql-parsing-domain-specific-languages/}
\newline

\url{http://unicode.org/glossary/}


Directory Services
\newline

\url{http://en.wikipedia.org/wiki/Directory_service}
\newline

\url{http://en.wikipedia.org/wiki/X.500}
Comparison of Directory Services Solutions
\newline

Microsoft Active Directory
\newline

Windows NT Directory Services
\newline

Novell eDirectory
\newline

Red Hat Directory Server
\newline

Apple Open Directory
\newline

Apache Directory Server
\newline

Oracle Internet Directory
\newline

CA Directory
\newline

Alcatel-Lucent Directory Server
\newline

Sun Java System Directory Server
\newline

OpenDS
\newline

IBM Tivoli Directory Server
\newline

Siemens DirX Directory Server
\newline

Critical Path Directory Server
\newline

OpenLDAP
\newline

Isode Limited
\newline

UnboundID Directory Server
\newline

Lotus Domino

LDAP Data Interchange Format
\newline

\url{http://en.wikipedia.org/wiki/LDAP_Data_Interchange_Format}
\newline

RFC 2849 -  The LDAP Data Interchange Format (LDIF) - Technical Specification
\newline

RFC 4510 - Lightweight Directory Access Protocol (LDAP): Technical Specification Road Map -           
RFC 4525 — LDAP Modify-Increment Extension
Directory Services Markup Language
\newline

\url{http://en.wikipedia.org/wiki/Directory_Service_Markup_Language}

Metadirectory
\newline
\url{http://en.wikipedia.org/wiki/MetaDirectory}

Virtual directory
\url{http://en.wikipedia.org/wiki/Virtual_directory}

Group policy
\newline
\url{http://en.wikipedia.org/wiki/Group_Policy}


